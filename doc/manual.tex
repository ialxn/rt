\documentclass[10pt,a4paper,titlepage]{article}

\newcommand{\rt}{{\tt rt} }

\title{ {\Huge Manual for \\
		\rt \\
		Version 0.1 \\
		\vspace{2cm}}}

\author{Dr. Ivo Alxneit \\
	Paul Scherrer Institute \\
	CH-5232 Villigen PSI \\
	SWITZERLAND \\
	\\
	phone: ++41 56 310 4092 \\
	fax: ++41 56 310 4416 \\
	e-mail: ivo.alxneit@psi.ch \\
	\vspace{4cm}}


\begin{document}

\maketitle

\pagenumbering{roman}
\tableofcontents
\pagebreak

\setcounter{page}{1}
\pagenumbering{arabic}



\section{Installation}

To compile and install \rt you need {\tt (I)make}, a C-compiler. In addition, the following libraries are needed: {\tt GSL} (Gnu Scientific Library), a C-version of BLAS (Basic Linear Algebra System) such as {\tt cblas} or {\tt Atlas}, and {\tt libconfig}. Please set (or modify) {\tt CCOPTIONS} in {\tt Imakefile}. Most probably you will make sure that {\tt -O2 -g } are set. Run {\tt xmkmf -a} to have {\tt imake} build a {\tt Makefile} from the {\tt Imakefile} provided. Run {\tt make} to compile and build the binaries \rt and {\rt-semistatic} (a semi statically linked version {\rt-semistatic}.

Currently you will get the following (harmless) warnings:

\begin{verbatim}
target_plane_screen.c: In function ‘ps_get_intercept’:
target_plane_screen.c:135: warning: unused parameter ‘r’
\end{verbatim}



\section{Command line parameters}

{\bf Syntax:} \rt [options] {\tt <} input {\tt >} output 2 {\tt >} errors
\vspace{1em}
\begin{list}{}
{\setlength{\leftmargin}{3.5cm}
\setlength{\labelwidth}{3.0cm}
\setlength{\rightmargin}{0.5cm}}

\item[\tt [--append|-a]] Append to output files. A new seed value for the random number generator is supplied on the command line. This overrides the keyword {\tt seed} in the input file.

\item[\tt [--mode|-m]] Select mode [012] for this run.
\begin{itemize}
\item[\tt -m0]{Check syntax of input [default]}
\item[\tt -m1]{Output {\tt OFF} files useful to visualize the geometry\footnote{geomview}}
\item[\tt -m2]{perform a ray traceing run}
\end{itemize}

\item[\tt [--help|-h]] prints a small help message.

\item[\tt [--Version|-V]] Prints the version number.

\end{list}


\section{Structure in input file}
Input is given by a series of 'keyword'='value' pairs using {\tt libconfig}\footnote{Version 1.4.6, www.xxx.yyy-}. Keywords and strings\footnote{in double quotes.} are case sensitive, float values must contain a decimal point. For details on the syntax of the configuration files, please consult the documentation of {\tt libconfig}. Note, that {\tt libconfig} treats everything after the \# character until the end of the line as comment, that {\tt libconfig} is white space agnostic, and that {\tt libconfig} allows up to ten levels of included files (via the {\tt @include "filename"} directive. The general structure of the input is as follows:

\begin{verbatim}
global_parameter_1 = value_1;
global_parameter_2 = value_2;
.
global_parameter_n = value_n;

sources = (
{definition of source_1},
{definition of source_2},
.
{definition of source_n}
);

targets = (
{definition of target_1},
{definition of target_2},
.
{definition of target_n},
);

\end{verbatim}

All sources and all targets are identified by their {\tt type}. The common {\tt name} keyword\footnote{must be unique.} is used as base name for the various output files\footnote{{\tt axes\_NAME.off} and {\tt NAME.off} (mode 1)  and {\tt NAME.dat} for targets only (mode 2).}. The order of the keywords is not important.

\subsection{Global Parameters}
\begin{itemize}
\item[{\bf seed}:]{Seed value (unsigned integer) for random number generator. Using the same seed value again will result in the identical sequence of pseudo random number.\\
The value will be overridden by the {\tt -a} command line argument.}
\end{itemize}



\subsection{Sources}

All sources define the keywords {\tt power} and {\tt n\_rays}, i.e.~the total power assigned to the source and the number of rays that start from it. Each ray thus represents a power of {\tt power}/{\tt n\_rays}.

\subsubsection{Uniform Point Source}

\noindent {\tt type="uniform point source";}

\noindent {\tt name="NAME";}

\noindent {\tt origin=[x,y,z];}

\noindent {\tt power=P.p;}

\noindent {\tt n\_rays=N;}

Defines an uniform point source located at {\tt origin}.


\subsubsection{Spot Source}

\noindent {\tt type="spot source";}

\noindent {\tt name="NAME,";}

\noindent {\tt origin=[x,y,z];}

\noindent {\tt direction=[x,y,z];}

\noindent {\tt theta=P.p;}

\noindent {\tt power=P.p;}

\noindent {\tt n\_rays=N;}

Defines a point source located at {\tt origin}. The source emits homogeneously into a cone pointing in direction {\tt direction}. {\tt theta} defines the angle between the edge of the cone and {\tt direction}. If {\tt theta=0.0} all rays are emitted in direction {\tt direction}. {\tt theta=180.0} defines an uniform point source.


\subsection{Targets}

\subsubsection{One-Side Plane Screen}

\noindent {\tt type="one-sided plane screen";}

\noindent {\tt name="NAME,";}

\noindent {\tt point=[x,y,z];}

\noindent {\tt normal=[x,y,z];}

\noindent {\tt x=[x,y,z];}

Defines a transparent and non-absorbing infinite plane. The plane is defined by a point {\tt point} and its surface normal {\tt normal}. {\tt x} defines the x axis of the local coordinate system. The y axis of the local, right-handed system follows from $y=z \times x$. The one-side plane screen registers all rays that intercept the plane anti-parallel to its normal vector.


\subsubsection{Two-Side Plane Screen}

\noindent {\tt type="two-sided plane screen";}

\noindent {\tt name="NAME,";}

\noindent {\tt point=[x,y,z];}

\noindent {\tt normal=[x,y,z];}

\noindent {\tt x=[x,y,z];}

Defines a transparent and non-absorbing infinite plane. The plane is defined by a point {\tt point} and its surface normal {\tt normal}. {\tt x} defines the x axis of the local coordinate system. The y axis of the local, right-handed system follows from $y=z \times x$. The two-side plane screen registers all rays that intercept the plane.


\subsubsection{Triangle}

\noindent {\tt type="triangle";}

\noindent {\tt name="NAME,";}

\noindent {\tt P1=[x,y,z];}

\noindent {\tt P2=[x,y,z];}

\noindent {\tt P3=[x,y,z];}

\noindent {\tt reflectivity=P.p;}

Defines a triangular reflecting surface. {\tt P1}, {\tt P2}, and {\tt P3} define the three vertices of the triangle. The surface normal of the triangle points towards the observer if {\tt P1}, {\tt P2}, and {\tt P3} are defined anti-clockwise. The surface normal\footnote{local z axis.} defined in this way points away from the reflective surface with reflectivity {\tt reflectivity}. The opposite surface is non-reflecting and absorbs all radiation. The origin of the local, right handed coordinate system is located at {\tt P1} and $P2 - P1$ defines the local x axis. 


\subsubsection{Square}

\noindent {\tt type="square";}

\noindent {\tt name="NAME,";}

\noindent {\tt P1=[x,y,z];}

\noindent {\tt P2=[x,y,z];}

\noindent {\tt P3=[x,y,z];}

\noindent {\tt reflectivity=P.p;}

Defines a rectangular reflecting surface of dimension $a \times b$. {\tt P1}, {\tt P2}, and {\tt P3}, the three vertices define the two perpendicular sides. $a=P2-P1$, is parallel to the local x axis and $b=P3 - P1$ is parallel to the local y axis. The local origin is at the center of the rectangle. The local z axis follows from the right handedness of the local system. The surface from which the z-axis points points away is reflecting while the opposite surface absorbs all radiation.

\subsubsection{Ellipsoid}

\noindent {\tt type="ellipsoid;"}

\noindent {\tt name="NAME,";}

\noindent {\tt center=[x,y,z];}

\noindent {\tt x=[x,y,z];}

\noindent {\tt z=[x,y,z];}

\noindent {\tt axes=[a,b,c];}

\noindent {\tt z\_min=P.p;}

\noindent {\tt z\_max=P.p;}

\noindent {\tt reflectivity=P.p;}

Defines an ellipsoid\footnote{${x^2}/{a^2}+{y^2}/{b^2}+{z^2}/{c^2}=1$} in the local system with origin at {\tt origin}. {\tt x} and {\tt z} are the directions of the local x and z axes, respectively. The semi axes of the ellipsoid AR given as {\tt axes}. The surface is only defined for $z_\mathrm{min} \leq z \leq z_\mathrm{max}$. Only the inside surface is reflecting while the outside surface absorbs all radiation.


\section{Availability}
\rt is free and available upon request from the author (ivo.alxneit@psi.ch).



\section{Bugs} Please send any bug report or fix or enhancement to the author. They probably will be include in a future version. However {\bf no} actual support will be available from the author.

\end{document}



