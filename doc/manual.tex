\documentclass[10pt,a4paper,titlepage]{article}
\usepackage{hyperref}
\usepackage{footmisc}
%\usepackage[T1]{fontenc}
%\usepackage{textcomp}

\include{version}
\providecommand{\rtVersion}{ none defined}
\newcommand{\rt}{{\tt rt} }

\title{ {\Huge Manual for \\
		\rt \\
		Version:\rtVersion \\
		\vspace{2cm}}}

\author{Dr. Ivo Alxneit \\
	Paul Scherrer Institute \\
	CH-5232 Villigen PSI \\
	SWITZERLAND \\
	\\
	phone: ++41 56 310 4092 \\
	fax: ++41 56 310 4416 \\
	e-mail: ivo.alxneit@psi.ch \\
	\vspace{4cm}}


\begin{document}

\maketitle

\pagenumbering{roman}
\tableofcontents
\pagebreak

\setcounter{page}{1}
\pagenumbering{arabic}



\section{Installation}

To compile and install \rt you need a C-compiler\footnote{\rt is only tested with gcc} and, ideally, {\tt (I)make}. In addition, the following libraries are needed: {\tt GSL}\footnote{Gnu Scientific Library, \url{http://www.gnu.org/software/gsl/}}, a C-version of BLAS (Basic Linear Algebra System) such as {\tt gslcblas} or {\tt Atlas}\footnote{\url{http://math-atlas.sourceforge.net/}}, {\tt libconfig}\footnote{Version 1.4.6+, \url{http://www.hyperrealm.com/libconfig/}}, and {\tt libpthread}. Please set (or modify) {\tt CCOPTIONS} in {\tt Imakefile}. Most probably, you will make sure that {\tt -O2 -g} is set. If you add {\tt -DDEBUG} to {\tt CCOPTIONS}, debug information is printed on {\tt stderr} for targets where the ray--target intercept cannot be found analytically. This is explicitly mentioned in the manual for all targets that fall into this category. Run {\tt xmkmf -a} to have {\tt imake} build a {\tt Makefile} from the {\tt Imakefile} provided. Run {\tt make} to compile and build the binaries.



\section{Command line parameters}

{\bf Syntax:} \rt [options] {\tt <} input {\tt >} output 2 {\tt >} errors
\vspace{1em}
\begin{list}{}
{\setlength{\leftmargin}{3.5cm}
\setlength{\labelwidth}{3.0cm}
\setlength{\labelsep}{0.25cm}
\setlength{\rightmargin}{0.5cm}}

\item[{\tt [--append|-a]}] Append to output files. A new seed value for the random number generator is supplied with this parameter. This overrides the keyword {\tt seed} in the input file. Use this option to improve existing results by tracing additional rays and append to existing output files. Default mode is to truncate existing files.

\item[{\tt [--keep\_closed|-k]}] For each target one output file is produced. Unless this option is given all these files are kept open. For very large system you might thus run out of file descriptors\footnote{Typically 1024 file descriptor can be used per process. Query with {\tt ulimit -n}}. If this option is given, all per target output files are kept closed and only opened when buffers are written to them.

\item[{\tt [--Log|-L]}] Log the path of some rays in raw format (give number of rays to be used as parameter) for each source in file {\tt NNN.dat} with {\tt NNN} denoting the number (ID) of the source. Output format is as follows:
\begin{verbatim}
x_start y_start z_start x_end y_end z_end
x_start y_start z_start x_end y_end z_end
.
\end{verbatim}
Start and end of each leg of the path is on a single line. Individual rays are separated by an empty line.  

\item[{\tt [--log|-l]}] Log the path of some rays in OFF format (give number of rays to be used as parameter) for each source in file {\tt NNN.OFF} with {\tt NNN} denoting the number (ID) of the source. Rays that continue to infinity after their last intercept are made 1.0 length unit long and painted red.

\item[{\tt [--mode|-m]}] Select mode [012] for this run.
\begin{itemize}
\item[\tt -m0]{Check syntax of input [default].}
\item[\tt -m1]{Output {\tt OFF} files useful to visualize the geometry\footnote{geomview availabe at \url{http://geomview.org}}.}
\item[\tt -m2]{Perform a ray tracing run.}
\end{itemize}

\item[{\tt [--threads|-t]}] {Set number of worker threads [1]. Typically, you want to set this parameter to the number of CPUs or cores available. Each worker thread uses its own rng seeded with {\tt seed}, {\tt seed}+1}, \ldots{}.

\item[{\tt [--help|-h]}] prints a help message.

\item[{\tt [--Version|-V]}] Prints the version number.

\end{list}


\section{Structure of the input file}
Input is given by a series of 'keyword'='value' pairs and parsed by {\tt libconfig}. Note, that each pair ends with ';', that keywords and strings\footnote{in double quotes.} are both case sensitive, and that float values \emph{must} contain a decimal point. {\tt libconfig} treats everything after the \# character until the end of the line as comment, is white space agnostic, and allows up to ten levels of included files (via the {\tt @include "filename"} directive. For details on the syntax used in the configuration files, please consult the documentation of {\tt libconfig}.
The general structure of the input --- the order of the keyword being unimportant --- is as follows:

\begin{verbatim}
global_parameter_1 = value_1;
global_parameter_2 = value_2;
.
global_parameter_n = value_n;

sources = (
{definition of source_1},
{definition of source_2},
.
{definition of source_n}
);

targets = (
{definition of target_1}, 
{definition of target_2},
.
{definition of target_n}
);
\end{verbatim}

All sources and all targets are identified by their {\tt type}. The common {\tt name} keyword\footnote{must be unique.} is used as base name for the various output files\footnote{{\tt axes\_NAME.off} and {\tt NAME.off} (mode 1) and {\tt NAME.dat} for targets only (mode 2).}.

\subsection{Global Parameters}
\begin{itemize}
\item[{\bf seed}:]{Seed value (unsigned integer) for random number generator. Using the same seed value again will result in the identical sequence of pseudo random number and thus an identical output.\footnote{This is not true anymore if you use more than one worker thread. First, each worker uses its own rng and seed value. Each produces a unique sequence of pseudo random numbers. Second, the exact sequence with which blocks of output from the different worker are interleaved in the final output is non-deterministic. Third, the amount of work performed by an individual worker is non-deterministic.}\\
The value will be overridden by the {\tt -a} command line argument. Using this option, existing results can be improved by tracing additional rays.}
\item[{\bf P\_factor}:]{Power represented by one (absorbed) ray. The number of rays to be tracked for each source is calculated as {\tt n\_rays} = {\tt power} / {\tt P\_factor}.}
\end{itemize}



\section{Sources}

All sources define the keywords {\tt origin} (doubles), {\tt power} (double), and {\tt spectrum} (string) i.e.\ the origin of the source, the total power assigned to the source, and the name of a file\footnote{The spectrum will be represented by a linear interpolation of the data given in this file. Use a minimum number of data points to make the interpolation fast} containing the spectrum of the source. Each ray thus represents a power of {\tt P\_factor} at a random wavelength.

\subsection{Transparent (non-interacting) Sources}

Rays that intercept these sources pass through them without being changed in any way.

\subsubsection{Arc}

\noindent {\tt type="arc";}

\noindent {\tt name="NAME";}

\noindent {\tt origin=[X.x, Y.y, Z.z];}

\noindent {\tt direction=[X.x, Y.y, Z.z];}

\noindent {\tt radius=R.r;}

\noindent {\tt length=L.l;}

\noindent {\tt power=P.p;}

\noindent {\tt spectrum="FILE\_NAME";}

\vspace{0.25cm}
Defines an cylindrical arc of radius {\tt radius} and length {\tt length} in direction {\tt direction} its base located at {\tt origin} .

\subsubsection{Uniform Point Source}

\noindent {\tt type="uniform point source";}

\noindent {\tt name="NAME";}

\noindent {\tt origin=[X.x, Y.y, Z.z];}

\noindent {\tt power=P.p;}

\noindent {\tt spectrum="FILE\_NAME";}

\vspace{0.25cm}
Defines an uniform point source located at {\tt origin}. 


\subsubsection{Spot Source}

\noindent {\tt type="spot source";}

\noindent {\tt name="NAME";}

\noindent {\tt origin=[X.x, Y.y, Z.z];}

\noindent {\tt direction=[X.x, Y.y, Z.z];}

\noindent {\tt theta=T.t;}

\noindent {\tt power=P.p;}

\noindent {\tt spectrum="FILE\_NAME";}

\vspace{0.25cm}
Defines a point source located at {\tt origin}. The source emits homogeneously into a cone pointing in {\tt direction}. {\tt theta} defines the angle between the edge of the cone and {\tt direction}. If {\tt theta=0.0} all rays are emitted in {\tt direction}. {\tt theta=180.0} defines an uniform point source. The direction vector {\tt direction} does not need to be normalized.

\subsubsection{Sphere}

\noindent {\tt type="sphere";}

\noindent {\tt name="NAME";}

\noindent {\tt origin=[X.x, Y.y, Z.z];}

\noindent {\tt radius=R.r;}

\noindent {\tt power=P.p;}

\noindent {\tt spectrum="FILE\_NAME";}

\vspace{0.25cm}

Defines an uniform spherical source of radius {\tt radius} located at {\tt origin}. Rays start at a random position inside the sphere in a random direction. Both, origin and direction of the ray are uniformly distributed.

\subsection{Solid (non-transparent, re-emitting) Sources}

Solid sources must have their reflectivity spectrum ({\tt reflectivity="FILE\_NAME"}) and a reflectivity model ({\tt reflectivity\_model="MODEL"}\footnote{see Sec.~\ref{sec:refl_models}}) defined. Rays that hit these sources can be reflected or absorbed. Absorbed ray are re-emitted again from this source as a random ray with the spectral characteristic of this source independently from their initial origin. Note, that also re-emission of a ray is counted as a reflection. Thus the number of reflection corresponds to the number of reflections and re-emission a ray experiences on his way from the \emph{original} sources until it is either absorbed or leaves the system.

\subsubsection{Solid Cone}

\noindent {\tt type="solid cone";}

\noindent {\tt name="NAME";}

\noindent {\tt origin=[X.x, Y.y, Z.z];}

\noindent {\tt z=[X.x, Y.y, Z.z];}

\noindent {\tt R=R.r;}

\noindent {\tt r=R.r;}

\noindent {\tt h=H.h;}

\noindent {\tt base\_face\_emits=[true|false];}

\noindent {\tt top\_face\_emits=[true|false];}

\noindent {\tt power=P.p;}

\noindent {\tt spectrum="FILE\_NAME";}


\vspace{0.25cm}
Defines a solid cone of height {\tt h} with its axis defined by {\tt z}. The center of its base face with radius {\tt R} is located at {\tt origin}. The radius of the top face is {\tt r}. Depending on the settings for {\tt base\_face\_emits} and {\tt top\_face\_emits}, rays start at a random position on the conical wall only or also on one or two of its covering faces. Both, origin and direction of the rays are uniformly distributed. Faces that do not emit should be covered\footnote{Use a tiny offset} by a separate target of type disk with properly chosen optical properties to account for rays intercepted by the non-emitting surface(s).

\subsubsection{Solid Cylinder}

\noindent {\tt type="solid cylinder";}

\noindent {\tt name="NAME";}

\noindent {\tt origin=[X.x, Y.y, Z.z];}

\noindent {\tt radius=R.r;}

\noindent {\tt direction=[X.x, Y.y, Z.z];}

\noindent {\tt length=L.l;}

\noindent {\tt base\_face\_emits=[true|false];}

\noindent {\tt top\_face\_emits=[true|false];}

\noindent {\tt power=P.p;}

\noindent {\tt spectrum="FILE\_NAME";}


\vspace{0.25cm}
Defines a solid cylinder of radius {\tt radius} and length {\tt length} with the center of its base face located at {\tt origin}. Depending on settings for {\tt base\_face\_emits} and {\tt top\_face\_emits}, rays start at a random position on the cylinder wall only or also on one or two of its covering faces. Both, origin and direction of the ray are uniformly distributed. Faces that do not emit should be covered\footnote{Use a tiny offset} by a separate target of type disk with properly chosen optical properties to account for rays intercepted by the non-emitting surface(s).

\subsubsection{Solid Sphere}

\noindent {\tt type="solid sphere";}

\noindent {\tt name="NAME";}

\noindent {\tt origin=[X.x, Y.y, Z.z];}

\noindent {\tt radius=R.r;}

\noindent {\tt power=P.p;}

\noindent {\tt spectrum="FILE\_NAME";}


\vspace{0.25cm}
Defines an uniform spherical source of radius {\tt radius} located at {\tt origin}. Rays start at a random position on the sphere in a random direction. Both, origin and direction of the ray are uniformly distributed.




\section{Targets}
All targets may use the optional keyword {\tt no\_output} ({\tt [true|false]}) that controls whether absorbed rays are logged for this target. The (implied) default is {\tt false} i.e.\ output is produced. All interacting\footnote{Targets that can absorb, refract, reflect, or scatter rays in contrast to {\it screens} that only register rays passing through them.} targets need the keywords {\tt reflectivity} (filename of spectral reflectivity) and {\tt reflectivity\_model} (see Sec.~\ref{sec:refl_models}) defined.

The structure of all output files (one for each target unless {\tt no\_output=True} is set) is identical: First, an ASCII header (lines starting with {\tt \#}) with name and type of the target, follwed by the transformation matrix $M$ and the origin $O$ required to transform from local to global coordinates. Then binary data sets (one per absorbed\footnote{or registered rays for screens} ray) follow consisting of 3\footnote{4, for non-planar targets} single precicions floats and one unsigned character that represent x, y, z\footnote{missing for planar targets}, $\lambda$, $n_{\mathrm{refl}}$.


\subsection{Screens}

The binary part of the output for screens consists of three single precision floats and one unsigned character per registered ray. The data represents $x, y, \lambda$, $n_{\mathrm{refl}}$.

\subsubsection{One-Sided Plane Screen}

\noindent {\tt type="one-sided plane screen";}

\noindent {\tt name="NAME";}

\noindent {\tt point=[X.x, Y.y, Z.z];}

\noindent {\tt normal=[X.x, Y.y, Z.z];}

\noindent {\tt x=[X.x, Y.y, Z.z];}

\vspace{0.25cm}
Defines a transparent and non-absorbing infinite plane. The plane is defined by a point {\tt point} and the direction of its surface normal {\tt normal} that does not need to be normalized and is used as local $z$ axis. {\tt x} defines the $x$ axis of the local coordinate system. The $y$ axis of the local, right-handed system follows from $y=z \times x$. The one-side plane screen registers all rays that intercept the plane anti-parallel to its normal vector.s


\subsubsection{Two-Sided Plane Screen}

\noindent {\tt type="two-sided plane screen";}

\noindent {\tt name="NAME";}

\noindent {\tt point=[X.x, Y.y, Z.z];}

\noindent {\tt normal=[X.x, Y.y, Z.z];}

\noindent {\tt x=[X.x, Y.y, Z.z];}

\vspace{0.25cm}
Defines a transparent and non-absorbing infinite plane. The plane is defined by a point {\tt point} and the direction of its surface normal {\tt normal} that does not need to be normalized and is used as local $z$ axis. {\tt x} defines the $x$ axis of the local coordinate system. The $y$ axis of the local, right-handed system follows from $y=z \times x$. In contrast the the one-side plane screen the two-side plane screen registers \emph{all} rays that intercept the plane independent of their direction.


\subsection{Planar targets}

The binary part of the output for screens consists of three single precision floats and one unsigned character per registered ray. The data represents $x, y, \lambda$, $n_{\mathrm{refl}}$ with $x$ and $y$ given in the local system i.e.\ $z$ is always zero and not logged.


\subsubsection{Annulus}

\noindent {\tt type="annulus";}

\noindent {\tt name="NAME";}

\noindent {\tt P=[X.x, Y.y, Z.z];}

\noindent {\tt N=[X.x, Y.y, Z.z];}

\noindent {\tt R=R.r;}

\noindent {\tt r=R.r;}

\noindent {\tt x=[X.x, Y.y, Z.z];}

\noindent {\tt reflectivity="FILE\_NAME";}

\noindent {\tt reflectivity\_model="MODEL";}\footnote{see Sec.~\ref{sec:refl_models}}

\vspace{0.25cm}
Defines a reflecting annulus of radius {\tt R} with a hole of radius {\tt r} at its center. The annulus is centered at {\tt P}, its orientation defined by its normal vector {\tt N}. The normal vector points away from the reflecting surface characterized by its reflectivity spectrum given by the file {\tt reflectivity} and the reflectivity model {\tt reflectivity\_model}. The backside of the annulus has implicit absorptivity of 1.0. {\tt x} defines the $x$ axis and {\tt N} the $z$ axis of the local coordinate system of the disk. If {\tt x} is not perpendicular to {\tt N} its perpendicular component (after normalization is used. The local $y$ axis is calculated from $y = N \times x$.

\subsubsection{Disk}

\noindent {\tt type="disk";}

\noindent {\tt name="NAME";}

\noindent {\tt P=[X.x, Y.y, Z.z];}

\noindent {\tt N=[X.x, Y.y, Z.z];}

\noindent {\tt r=R.r;}

\noindent {\tt x=[X.x, Y.y, Z.z];}

\noindent {\tt reflectivity="FILE\_NAME";}

\noindent {\tt reflectivity\_model="MODEL";}\footnote{see Sec.~\ref{sec:refl_models}}

\vspace{0.25cm}
Defines a reflecting disk of radius {\tt r} with center at {\tt P} and normal vector {\tt N}. The normal vector serves as local z axis and point away from the reflecting surface. The backside of the disk has implicit absorptivity of 1.0. The spectral reflectivity of the front surface is defined in the file {\tt reflectivity} and the reflectivity model {\tt reflectivity\_model} is applied. {\tt x} defines the x axis and {\tt N} the $z$ axis of the local coordinate system of the disk. If {\tt x} is not perpendicular to {\tt N} its perpendicular component (after normalization is used. The local $y$ axis is calculated from $y = N \times x$.

\subsubsection{Rectangle}

\noindent {\tt type="rectangle";}

\noindent {\tt name="NAME";}

\noindent {\tt P1=[X.x, Y.y, Z.z];}

\noindent {\tt P2=[X.x, Y.y, Z.z];}

\noindent {\tt P3=[X.x, Y.y, Z.z];}

\noindent {\tt reflectivity="FILE\_NAME";}

\noindent {\tt reflectivity\_model="MODEL";}\footnote{see Sec.~\ref{sec:refl_models}}

\vspace{0.25cm}
Defines a rectangular reflecting surface of dimension $a \times b$. {\tt P1}, {\tt P2}, and {\tt P3}, the three vertices define the two perpendicular sides. $a=$ {\tt P2} - {\tt P1}, is parallel to the local $x$ axis and $b=$ {\tt P3} - {\tt P1} is parallel to the local y axis. The local origin is at the center of the rectangle. The local $z$ axis follows from the right handedness of the local system. The surface from which the $z$ axis points away is reflecting characterized by its reflectivity spectrum given by the file {\tt reflectivity} and the reflectivity model {\tt reflectivity\_model} while the other surface absorbs all rays that impinge on it.

\subsubsection{Triangle}

\noindent {\tt type="triangle";}

\noindent {\tt name="NAME";}

\noindent {\tt P1=[X.x, Y.y, Z.z];}

\noindent {\tt P2=[X.x, Y.y, Z.z];}

\noindent {\tt P3=[X.x, Y.y, Z.z];}

\noindent {\tt reflectivity="FILE\_NAME";}

\noindent {\tt reflectivity\_model="MODEL";}\footnote{see Sec.~\ref{sec:refl_models}}

\vspace{0.25cm}
Defines a triangular reflecting surface. {\tt P1}, {\tt P2}, and {\tt P3} define the three vertices of the triangle. The surface normal of the triangle points towards the observer if {\tt P1}, {\tt P2}, and {\tt P3} are defined anti-clockwise. The surface normal\footnote{local $z$ axis.} defined in this way points away from the reflective surface characterized by its reflectivity spectrum given by the file {\tt reflectivity} and the reflectivity model {\tt reflectivity\_model}. The other surface is non-reflecting and absorbs all rays that impinge on it. The origin of the local, right handed coordinate system is located at {\tt P1} and {\tt P2} - {\tt P1} defines the local $x$ axis. 

\subsubsection{Window}

\noindent {\tt type="window";}

\noindent {\tt name="NAME";}

\noindent {\tt C=[X.x, Y.y, Z.z];}

\noindent {\tt a=[X.x, Y.y, Z.z];}

\noindent {\tt d=D.d;}

\noindent {\tt r=R.r;}

\noindent {\tt x=[X.x, Y.y, Z.z];}

\noindent {\tt idx\_refraction="FILE\_NAME";}

\noindent {\tt absorptivity="FILE\_NAME";}

\noindent {\tt reflectivity\_model="MODEL";}\footnote{see Sec.~\ref{sec:refl_models}}

\vspace{0.25cm}
Defines a transparent, round window of radius {\tt r} and thickness {\tt d}. The center of the first face is at {\tt C} while the second face is offset by {\tt d} in direction {\tt a}. The window material is characterized by the index of refraction {\tt idx\_refraction} and the spectral absorptivity {\tt absorptivity} \footnote{per length unit}. The reflectivity model {\tt reflectivity\_model} is applied at both faces. At present only the reflectivity model {\tt "specular"} is supported. For the cylindical wall $a=1$ is assumend. {\tt x} defines the local $x$ axis, the local $y$ axis is calculated from $y = a \times x$.



\subsection{Non-planar targets}

The binary part of the output for screens consists of four single precision floats and one unsigned character per registered ray. The data represents $x, y, z, \lambda$, and $n_{\mathrm{refl}}$ with $x, y,$ and $z$ given in the local system. All non-planar targets have to have the keyword {\tt reflecting\_surface}\footnote{{\tt INSIDE} (concave side) or {\tt OUTSIDE} (convex side)} defined to identifiy the reflecting surface.

\subsubsection{Cone}

\noindent {\tt type="cone";}

\noindent {\tt origin=[X.x, Y.y, Z.z];}

\noindent {\tt axis=[X.x, Y.y, Z.z];}

\noindent {\tt R=R.r;}

\noindent {\tt r=R.r;}

\noindent {\tt h=H.h;}

\noindent {\tt x=[X.x, Y.y, Z.z];}

\noindent {\tt reflectivity="FILE\_NAME";}

\noindent {\tt reflectivity\_model="MODEL";}\footnote{see Sec.~\ref{sec:refl_models}}

\noindent {\tt reflecting\_surface=["inside"|"outside"];}

\vspace{0.25cm}
Defines a reflecting cone of radius {\tt R} at its base and radius {\tt r} at height {\tt h}. The center of the base is at {\tt origin}. The orientation of the cone is define by its symmetry axis {\tt axis}\footnote{not necessarily normalized} that also serves as local $z$ axis. The reflecting surface identified by {\tt reflecting\_surface} is characterized by the reflectivity spectrum given by the file {\tt reflectivity} and the reflectivity model {\tt reflectivity\_model}. The other surface has implicit absorptivity of 1.0. {\tt x} defines the $x$ axis of the local coordinate system of the cone. The local $y$ axis is calculated from $y = N \times x$.

\subsubsection{(truncated) CPC}

\noindent {\tt type="cpc";}

\noindent {\tt origin=[X.x, Y.y, Z.z];}

\noindent {\tt axis=[X.x, Y.y, Z.z];}

\noindent {\tt acceptance\_angle=A.a;}

\noindent {\tt truncation\_angle=T.t;}

\noindent {\tt exit\_radius=R.r;}

\noindent {\tt x=[X.x, Y.y, Z.z];}

\noindent {\tt reflectivity="FILE\_NAME";}

\noindent {\tt reflectivity\_model="MODEL";}\footnote{see Sec.~\ref{sec:refl_models}}

\vspace{0.25cm}
Defines a truncated CPC (compound parabolic concentrator) with an exit radius {\tt exit\_radius}. {\tt origin} defines the center of the exit aperture and the symmetry axis of the CPC is given by {\tt axis}\footnote{not necessarily normalized} that also serves as local $z$ axis. {\tt acceptance\_angle} \footnote{degrees versus symmetry axis of CPC.} defines the maximum angle where ray entering the CPC still exit by the exit aperture. {\tt truncation\_angle}\footnote{degrees versus symmetry axis of CPC. Must be larger than {\tt acceptance\_angle}.}. The reflecting surface is characterized by the reflectivity spectrum given by the file {\tt reflectivity} and the reflectivity model {\tt reflectivity\_model}. {\tt x} defines the $x$ axis of the local coordinate system of the cone. The local $y$ axis is calculated from $y = N \times x$. Note, that {\tt reflecting\_surface} is not needed for this target but that the user must ensure that no ray hits the CPC on its outside surface! Any ray that would be intercepted by the outside surface of the CPC passes through it. Also note, that no other target may reach into the CPC. Both restrictions are due to an efficient implementation.

No analytical solution exists for ray--target intercept. If {\tt -DDEBUG} is set in {\tt CCOPTIONS} debug output is printed on {\tt stderr}.


\subsubsection{Cylinder}

\noindent {\tt type="cylinder";}

\noindent {\tt C=[X.x, Y.y, Z.z];}

\noindent {\tt a=[X.x, Y.y, Z.z];}

\noindent {\tt r=R.r;}

\noindent {\tt l=L.l;}

\noindent {\tt x=[X.x, Y.y, Z.z];}

\noindent {\tt reflectivity="FILE\_NAME";}

\noindent {\tt reflectivity\_model="MODEL";}\footnote{see Sec.~\ref{sec:refl_models}}

\noindent {\tt reflecting\_surface=["inside"|"outside"];}

\vspace{0.25cm}
Defines a reflecting cylinder of radius {\tt r} and length {\tt l}. The center of the first face is at {\tt C}. The orientation of the cylinder is define by its axis vector {\tt a}\footnote{not necessarily normalized} pointing towards its second face. The reflecting surface identified by {\tt reflecting\_surface} is characterized by the reflectivity spectrum given by the file {\tt reflectivity} and the reflectivity model {\tt reflectivity\_model}. The other surface has implicit absorptivity of 1.0. {\tt x} defines the $x$ axis and {\tt a} the $z$ axis of the local coordinate system of the cylinder. The local $y$ axis is calculated from $y = N \times x$.

\subsubsection{Ellipsoid}

\noindent {\tt type="ellipsoid";}

\noindent {\tt name="NAME";}

\noindent {\tt center=[X.x, Y.y, Z.z];}

\noindent {\tt x=[X.x, Y.y, Z.z];}

\noindent {\tt z=[X.x, Y.y, Z.z];}

\noindent {\tt axes=[A.a, B.b, C.c];}

\noindent {\tt z\_min=I.i;}

\noindent {\tt z\_max=A.a;}

\noindent {\tt reflectivity="FILE\_NAME";}

\noindent {\tt reflectivity\_model="MODEL";}\footnote{see Sec.~\ref{sec:refl_models}}

\noindent {\tt reflecting\_surface=["inside"|"outside"];}

\vspace{0.25cm}
Defines an ellipsoid\footnote{${x^2}/{a^2}+{y^2}/{b^2}+{z^2}/{c^2}=1$} in the local system with origin at {\tt origin}. {\tt x} and {\tt z} are the directions of the local $x$ and $z$ axes, respectively. The semi axes of the ellipsoid are given as {\tt axes}. The surface is only defined for $z_\mathrm{min} \leq z \leq z_\mathrm{max}$. The reflecting surface identified by {\tt reflecting\_surface} is characterized by the reflectivity spectrum given by the file {\tt reflectivity} and the reflectivity model {\tt reflectivity\_model}. The other surface has implicit absorptivity of 1.0.

\subsubsection{Paraboloid}

\noindent {\tt type="paraboloid";}

\noindent {\tt name="NAME";}

\noindent {\tt vertex=[X.x, Y.y, Z.z];}

\noindent {\tt x=[X.x, Y.y, Z.z];}

\noindent {\tt z=[X.x, Y.y, Z.z];}

\noindent {\tt focal\_length=F.f;}

\noindent {\tt z\_min=I.i;}

\noindent {\tt z\_max=A.a;}

\noindent {\tt reflectivity="FILE\_NAME";}

\noindent {\tt reflectivity\_model="MODEL";}\footnote{see Sec.~\ref{sec:refl_models}}

\noindent {\tt reflecting\_surface=["inside"|"outside"];}

\vspace{0.25cm}
Defines an paraboloid\footnote{${x^2}/{4f}+{y^2}/{4f}-z=0$} in the local system with origin at the vertex {\tt vertex} of the paraboloid. {\tt x} and {\tt z} are the directions of the local $x$ and $z$ axes, respectively. The focal length $f$ is given as {\tt focal\_length}. The surface is only defined for $z_\mathrm{min} \leq z \leq z_\mathrm{max}$. The reflecting surface identified by {\tt reflecting\_surface} is characterized by the reflectivity spectrum given by the file {\tt reflectivity} and the reflectivity model {\tt reflectivity\_model}. The other surface has implicit absorptivity of 1.0.

\subsubsection{Sphere}

\noindent {\tt type="sphere";}

\noindent {\tt name="NAME";}

\noindent {\tt origin=[X.x, Y.y, Z.z];}

\noindent {\tt x=[X.x, Y.y, Z.z];}

\noindent {\tt z=[X.x, Y.y, Z.z];}

\noindent {\tt radius=R.r;}

\noindent {\tt z\_min=I.i;}

\noindent {\tt z\_max=A.a;}

\noindent {\tt reflectivity="FILE\_NAME";}

\noindent {\tt reflectivity\_model="MODEL";}\footnote{see Sec.~\ref{sec:refl_models}}

\noindent {\tt reflecting\_surface=["inside"|"outside"];}

\vspace{0.25cm}
Defines an sphere\footnote{${x^2}+{y^2}+{z^2}={r^2}$} with radius {\tt radius} centered at {\tt origin}. {\tt x} and {\tt z} are the directions of the local $x$ and $z$ axes, respectively. The surface is only defined for $z_\mathrm{min} \leq z \leq z_\mathrm{max}$ allowing to define spherical mirrors. The reflecting surface identified by {\tt reflecting\_surface} is characterized by the reflectivity spectrum given by the file {\tt reflectivity} and the reflectivity model {\tt reflectivity\_model}. The other surface has implicit absorptivity of 1.0.



\section{Reflectivity Models} \label{sec:refl_models}

Additional, model specific parameters might be defined by the different models.

\subsection{Specular Reflection}

\noindent {\tt reflectivity\_model="specular";} defines a perfectly specularly reflecting surface.

\subsection{Diffuse Reflection}

\noindent {\tt reflectivity\_model="lambertian";} defines a perfectly diffusly\footnote{Lambertian} reflecting surface.

\subsection{Microfacetted Surface}

\noindent {\tt reflectivity\_model="microfacet\_gaussian";} defines a surface consisting of randomly oriented micro facets. The normal vectors of the individual micro facets are gaussian (with width {\tt microfacet\_gaussian\_sigma} as degree) distriSbuted around the macroscopic surface normal.



\section{Utilities}

The structure of all output files (one for each target unless {\tt no\_output=True} is set) is identical: First, an ASCII header (lines starting with {\tt \#}) with name and type of the target, follwed by the transformation matrix $M$ and the origin $O$ required to transform from local to global coordinates. Then binary data sets (one per absorbed ray) follow consisting of 3\footnote{4, for non-planar targets} single precicions floats and one unsigned character (x, y, z\footnote{missing for planar targets}, $\lambda$, $n_{\mathrm{refl}}$) per absorbed ray.   

\subsection{{\tt spectrum}}

{\tt spectrum} reads the output file of a target produced by \rt and extract the spectrum (histogram) of the radiation absorbed by the target\footnote{or impinging on the target in case of a non-absorbing screen}. Statistical data of the spectrum are reported in its header including the number of rays and their total power that fall outside the range considered.

{\bf Syntax:} {\tt spectrum} [options] {\tt <} input {\tt >} output 2 {\tt >} errors
\vspace{1em}
\begin{list}{}
{\setlength{\leftmargin}{3.5cm}
\setlength{\labelwidth}{3.0cm}
\setlength{\labelsep}{0.25cm}
\setlength{\rightmargin}{0.5cm}}

\item[{\tt [--num|-n]}] Number of bins used [10].
\item[{\tt [--start|-a]}] Start wavelength for spectrum [0.0].
\item[{\tt [--stop|-o]}] Stop wavelength for spectrum [2000.0].
\item[{\tt [--help|-h]}] prints a help message.
\item[{\tt [--Version|-V]}] Prints the version number.

\end{list}

\subsection{{\tt flux\_2D}}
{\tt flux\_2D} outputs the distribution of the absorbed flux in the wavelength interval $ \lambda_\mathrm{min} \le \lambda \le \lambda_\mathrm{min}$ (2D histogram) of planar targets. Statistical data including the number of rays missed is reported in the header.

{\bf Syntax:} {\tt flux\_2D} [options] {\tt <} input {\tt >} output 2 {\tt >} errors
\vspace{1em}
\begin{list}{}
{\setlength{\leftmargin}{3.5cm}
\setlength{\labelwidth}{3.0cm}
\setlength{\labelsep}{0.25cm}
\setlength{\rightmargin}{0.5cm}}

\item[{\tt [--nx|-a]}] Number of bins in x direction used [10].
\item[{\tt [--ny|-b]}] Number of bins in y direction used [10].
\item[{\tt [--global|-g]}] report flux distribution in global coordinate system [local].
\item[{\tt [--minl|-l]}] Minimum wavelength [0.0].
\item[{\tt [--maxl|-L]}] Maximum wavelength [1E308].
\item[{\tt [--minx|-x]}] Minimum x value of histogram [-10.0].
\item[{\tt [--maxx|-X]}] Maximum x value of histogram [10.0].
\item[{\tt [--miny|-y]}] Minimum y value of histogram [-10.0].
\item[{\tt [--maxy|-Y]}] Maximum y value of histogram [10.0].
\item[{\tt [--help|-h]}] prints a help message.
\item[{\tt [--Version|-V]}] Prints the version number.

\end{list}

\subsection{{\tt flux\_3D.py}}
{\tt flux\_3D}\footnote{Requires {\tt python3}, {\tt numpy}, {\tt pandas}, and {\tt matplotlib}} outputs the distribution of the absorbed flux for non-planar targets (at present: {\tt cone}, {\tt cylinder}, {\tt paraboloid}, and {\tt sphere}).

{\bf Syntax:} {\tt flux\_3D.py} [options] {\tt <} input
\vspace{1em}
\begin{list}{}
{\setlength{\leftmargin}{3.5cm}
\setlength{\labelwidth}{3.2cm}
\setlength{\labelsep}{0.25cm}
\setlength{\rightmargin}{0.5cm}}

\item[{\tt [--nZ|-z]}] Number of bins in z direction used [10].
\item[{\tt [--nTheta|-a]}] Number of bins in $\Theta$ direction used [36].
\item[{\tt [--Zlimits|-Z]}] {\tt z\_min} and {\tt z\_max} of body (same as in config file).
\item[{\tt [--FLuxlimits|-F]}] Minimum and Maximum of flux scale.
\item[{\tt [--type|-t]}] Type of body ['cone'$\mid$'cylinder'$\mid$'paraboloid'$\mid$'sphere'].
\item[{\tt [--rlimits|-r]}] Radius at base and at top of cone.
\item[{\tt [--foc|-f}] Focal length of paraboloid.
\item[{\tt [--Pfactor|-P]}] Scaling factor of flux values (from config file).
\item[{\tt [--output|-o]}] Output to file. Format deduced from files extension.
\item[{\tt [--help|-h]}] prints a help message.

\end{list}

\subsection{{\tt get\_flux}}
{\tt get\_flux} extracts the binary output datafiles producted by {\tt rt} and lists the data. It outputs\footnote{tab separated} $x$ $y$ $\lambda$ $n_{\mathrm{refl}}$ for planar targets or $x$ $y$ $z$ $\lambda$ $n_{\mathrm{refl}}$ otherwise and only considers the wavelength interval the wavelength interval $ \lambda_\mathrm{min} \le \lambda \le \lambda_\mathrm{min}$.

{\bf Syntax:} {\tt get\_flux} [options] {\tt <} input {\tt >} output 2 {\tt >} errors
\vspace{1em}
\begin{list}{}
{\setlength{\leftmargin}{3.5cm}
\setlength{\labelwidth}{3.0cm}
\setlength{\labelsep}{0.25cm}
\setlength{\rightmargin}{0.5cm}}

\item[{\tt [--global|-g]}] report flux distribution in global coordinate system [local].
\item[{\tt [--minl|-l]}] minimum wavelength [0.0].
\item[{\tt [--maxl|-L]}] maximum wavelength [1E308].
\item[{\tt [--help|-h]}] prints a help message.
\item[{\tt [--Version|-V]}] Prints the version number.

\end{list}


\section{Availability}
\rt is free and available upon request from the author (ivo.alxneit@psi.ch).



\section{Bugs} Please send any bug report or fix or enhancement to the author. They probably will be included in a future version. However, {\bf no} actual support will be available from the author.

\end{document}



